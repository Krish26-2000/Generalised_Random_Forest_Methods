\documentclass[11pt, a4paper, leqno]{article}
\usepackage{a4wide}
\usepackage[T1]{fontenc}
\usepackage[utf8]{inputenc}
\usepackage{float, afterpage, rotating, graphicx}
\usepackage{epstopdf}
\usepackage{longtable, booktabs, tabularx}
\usepackage{fancyvrb, moreverb, relsize}
\usepackage{eurosym, calc}
% \usepackage{chngcntr}
\usepackage{amsmath, amssymb, amsfonts, amsthm, bm}
\usepackage{caption}
\usepackage{mdwlist}
\usepackage{xfrac}
\usepackage{setspace}
\usepackage[dvipsnames]{xcolor}
\usepackage{subcaption}
\usepackage{minibox}
% \usepackage{pdf14} % Enable for Manuscriptcentral -- can't handle pdf 1.5
% \usepackage{endfloat} % Enable to move tables / figures to the end. Useful for some
% submissions.

\usepackage[
    natbib=true,
    bibencoding=inputenc,
    bibstyle=authoryear-ibid,
    citestyle=authoryear-comp,
    maxcitenames=3,
    maxbibnames=10,
    useprefix=false,
    sortcites=true,
    backend=biber
]{biblatex}
\AtBeginDocument{\toggletrue{blx@useprefix}}
\AtBeginBibliography{\togglefalse{blx@useprefix}}
\setlength{\bibitemsep}{1.5ex}
\addbibresource{../../paper/refs.bib}

\usepackage[unicode=true]{hyperref}
\hypersetup{
    colorlinks=true,
    linkcolor=black,
    anchorcolor=black,
    citecolor=NavyBlue,
    filecolor=black,
    menucolor=black,
    runcolor=black,
    urlcolor=NavyBlue
}


\widowpenalty=10000
\clubpenalty=10000

\setlength{\parskip}{1ex}
\setlength{\parindent}{0ex}
\setstretch{1.5}


\begin{document}

\title{Generalised Random Forest Methods\thanks{Krishna Akolkar, University of Bonn. Email: \href{mailto:s6krakol@uni-bonn.de}{\nolinkurl{s6krakol [at] uni-bonn [dot] de}}.}}

\author{Krishna Akolkar}

\date{
    {\bf Preliminary -- please do not quote}
    \\[1ex]
    \today
}

\maketitle


\begin{abstract}
    This project is an attempt to use Machine learning methods like Generalized Random Forest for asserting the hypothesis that "After the removal of
    ban on same-sex marriages in the USA in from 2004- 2015, the same-sex married couples have increased". The extention of this study is to understand
    whether such couples are encouraged to start a family. The motivation behind this study is a research paper by Susan Athey, Julie Tibshirani and
    Stefan Wager on \href{https://arxiv.org/pdf/1610.01271}{Generalized Random Forest}.I have used the methods discussed in the paper and tried to implement
    the methods on a different dataset. The aim of both the studies is to study the heterogeneous treatment effects on the dependent variable. I have used
    the EconML package by Microsoft to implemement Causal Forest Double Machine Learning on the dataset. With this approach, I calculate the treatment effects
    on the outcome and ultimately get the average treatment effect. A positive conditional average treatment effect stats that the policy intervention or in
    our case, the removal of ban has a positive impact on the number of marriages in same-sex couples.
\end{abstract}

\clearpage


\section{Introduction} % (fold)
\label{sec:introduction}

If you are using this template, please cite this item from the references:
\citet{GaudeckerEconProjectTemplates}.

The data set for the example project is taken from
\url{https://www.stem.org.uk/resources/elibrary/resource/28452/large-datasets-stats4schools}.
It contains data on smoking habits in the UK, with 1691 observations and 12 variables.
We consider only 4 of the 12 features for the prediction of the variable
\texttt{smoking}: \texttt{marital\_status}, \texttt{highest\_qualification},
\texttt{gender} and \texttt{age}. We model the dependence using a Logistic model. All
numerical features are included linearly, while categorical features are expanded into
dummy variables. Figures below illustrate the model predictions over the lifetime. You
will find one figure and one estimation summary table for each installed programming
language.



\begin{figure}[H]

    \centering
    \includegraphics[width=0.85\textwidth]{../python/figures/smoking_by_marital_status}

    \caption{\emph{Python:} Model predictions of the smoking probability over the
        lifetime. Each colored line represents a case where marital status is fixed to one
        of the values present in the data set.}
    \label{fig:python-predictions}

\end{figure}


\begin{table}[!h]
    \input{../python/tables/estimation_results.tex}
    \caption{\label{tab:python-summary}\emph{Python:} Estimation results of the
        linear Logistic regression.}
\end{table}




% section introduction (end)



\setstretch{1}
\printbibliography
\setstretch{1.5}


% \appendix

% The chngctr package is needed for the following lines.
% \counterwithin{table}{section}
% \counterwithin{figure}{section}

\end{document}
